% -*- TeX:de -*-
\NeedsTeXFormat{LaTeX2e}
\documentclass[12pt,a4paper,titlepage]{article}

%\usepackage[german]{babel} % german text
\usepackage[DIV12]{typearea} % size of printable area
\usepackage[T1]{fontenc} % font encoding
\usepackage[utf8]{inputenc} % probably on Linux

\usepackage{graphicx} % to include images
\graphicspath{ {img/} } % set default image directory
\usepackage{subfigure} % for creating subfigures
\usepackage{amsmath} % a bunch of symbols
\usepackage{amssymb} % even more symbols
\usepackage{booktabs} % pretty tables
\usepackage{csquotes}

% a floating environment for circuits
\usepackage{float}
\usepackage{caption}

\newfloat{circuit}{tbph}{circuits}
\floatname{circuit}{Schaltplan}

% a floating environment for diagrams
\newfloat{diagram}{tbph}{diagrams}
\floatname{diagram}{Diagramm}

\renewcommand{\familydefault}{\sfdefault} % activate to use sans-serif font as default

\sloppy % friendly typesetting

\usepackage{eurosym}
\usepackage{makeidx}
\usepackage{amsfonts}
\usepackage{mparhack}
\usepackage{array}
\usepackage{tabularx}
\usepackage{minitoc}
\usepackage[colorlinks=true]{hyperref}
\usepackage{epstopdf}
\usepackage{setspace}
\usepackage{csquotes}
\usepackage{circuitikz}

% hyperref settings
\hypersetup{
    colorlinks=false,       % false: boxed links; true: colored links
    linkcolor=black,          % color of internal links (change box color with linkbordercolor)
    citecolor=black,        % color of links to bibliography
    filecolor=black,      % color of file links
    urlcolor=black           % color of external links
}

\begin{document}

\begin{titlepage}

\begin{figure*}[h!]
  \includegraphics[width=8cm]{TULogo_CMYK}
\end{figure*}

\begin{center}
\vspace*{1.3cm}
{\Huge Elektrotechnische Grundlagen der Informatik\\(LU 182.692)\\}
\vspace{1.7cm}
{\LARGE Protokoll der 2. Labor\"ubung: \enquote{Filter}\\}
{\large \enquote{Transiente Vorg\"ange und Frequenzverhalten}\\}
{\LARGE b) Messungen\\}
\vspace{1.5cm}

% fill in group number and date of lab here
% CHANGE ME!
{\Large Gruppennr.: 22 \hspace{1cm} Datum der Labor\"ubung: 19.05.2017}

% fill in IDs and names here
% CHANGE ME!
\begin{table}[h!]
\centering
\begin{tabular}{|p{3.5cm}|p{3.5cm}|p{6.5cm}|}
\hline \textbf{Matr. Nr.} & \textbf{Kennzahl} & \textbf{Name} \\
\hline
1614835 & 033 535 & Jan Nausner \\
\hline
1633068 & 033 535 & David Pernerstorfer \\
\hline
\end{tabular}
\end{table}

\end{center}
\vspace{1.0cm}

\begin{table}[h!]
\begin{tabular}{|l|l|}
\hline \textbf{Kontrolle} & \checkmark \\
\hline Verhalten eines RC-Filters 1. Ordnung & \\
\hline Verhalten eines RL-Filters 1. Ordnung & \\
\hline Dynamisches System 2. Ordnung & \\
\hline
\end{tabular}
\end{table}

\end{titlepage}
% start of actual lab protocol
% CHANGE ME!

\setcounter{page}{2}

\newpage
\setcounter{tocdepth}{1}
\tableofcontents

\newpage

\section*{Materialien}
\begin{itemize}
	\item Oszilloskop: Agilent InfiniiVision MSO-X 3054A
	\item Frequenzgenerator: Agilent 33220A
  \item Multimeter: Amprobe 37XR-A
\end{itemize}

\section{Messung des Verhaltens eines RC-Filters 1. Ordnung}

\subsection{Aufgabenstellung}
Die charakteristischen Eigenschaften eines RC-Filters 1. Ordnung mit realen Bauteilen sollen untersucht werden und mit der Simulation verglichen werden.

\subsection{Schaltplan}
\begin{figure}[H]
\centering
\begin{circuitikz}[european]
  \draw
    (0,2) to [R, l=$R$, o-] (3,2);
  \draw
    (3,2) to [short, *-o] (4,2);
  \draw
    (3,2) to [C, l=$C$] (3,0);
  \draw
    (0,0) to [short, o-o] (4,0);
  \draw
    (0,2) to [open, v=$U_e$] (0,0);
  \draw
    (4,2) to [open, v^=$U_a$] (4,0);
\end{circuitikz}
\caption{RC Tiefpassfilter 1.Ordnung}
\label{Figure01}
\end{figure}


\subsection{Durchf\"uhrung}
Die Schaltung wurde gem\"a\ss\, Schaltplan mit einem Widerstand $R=22k\Omega$ und einem Kondensator $C=10nF$ aufgebaut. Um die Sprungantwort aufzuzeichnen wurde am Eingang ein periodisches Rechtecksignal mit $1 V_{PP}$, Offset $0,5 V$, $50 \%$ Duty Circle und Frequenz $250 Hz$, angelegt. Die Eingangs- und Ausgangsspannung wurden mit dem Oszilloskop im Zeitbereich aufgezeichnet (siehe Abbildung~\ref{Figure02}). Die Zeitkonstante wurde mit den gemessenen Bauteilwerten ($R=21,75 k\Omega$, $C=10,28nF$) berechnet und ergibt einen Wert von $\tau = RC = 226,16 \mu s$. Die Zeitkonstante wurde aus der Sprungantwort ausgelesen indem der Zeitpunkt gemessen wurde, an dem das Signal der Sprungantwort ca. $63 \%$ der Eingangsspannung betrug (Vergleich siehe Tabelle ~\ref{Figure03}). \\
Am Eingang wurde nun ein Sinussignal mit $1 V_{PP}$, Offset $0 V$ angelegt. Die Eingangs- und Ausgangsspannung bzw. die Phasenverschiebung wurden mit dem Oszilloskop gemessen. Der Frequenzbereich von $10Hz$ bis $1MHz$ mit 5 Frequenzmesspunkten pro Dekade wurden aufgezeichnet. Weiters wurde ein Frequenzmesspunkt an der Grenzfrequenz gesetzt. Die Messpunkte wurden in tabellarischer Form aufgezeichnet und als Bodediagramm dargestellt (siehe Abbildung~\ref{Figure04} und ~\ref{Figure05}). Ab der Frequenz von rund $20kHz$ konnte die Phasenverschiebung nicht mehr gemessen werden, da der Messwert am Oszilloskop sehr schwach war und dadurch zu stark schwankte.

\subsection{Ergebnis \& Diskussion}
\begin{figure}[H]
  \centering
  \includegraphics[width=100mm]{sprungantwort_rc_tiefpass.png}
  \caption{Sprungantwort RC Tiefpass}
  \label{Figure02}
\end{figure}
\noindent In der Abbildung~\ref{Figure02} ist zu sehen, wie sich die Ausgangsspannung bei Anlegen eines Rechtecksignals verh\"alt. Ein Sprung am Eingang bewirkt am Ausgang eine Kurve die sich der Eingangspannung ann\"ahert. Das l\"asst auf ein System mit Energiespeicher schlie\ss en. In diesem Fall handelt es sich um einen Kondensator, weil anfangs die Spannung $0V$ ist und mit der Zeit die Impedanz des Kondensators steigt und somit auch die Spannung am Ausgang.

\begin{table}[H]
  \centering
  \begin{tabular}{|c|c|c|c|}
  \hline
                     & berechnet    & gemessen    & Abweichung \\ \hline
  Zeitkonstante $\tau$ & $226,16 \mu s$    & $230 \mu s$      & $+3,83 \mu s(+1,70 \%)$ \\ \hline
  \end{tabular}
  \caption{Vergleich Zeitkonstante $\tau$ berechnet und bemessen}
  \label{Figure03}
\end{table}
\noindent Die Zeitkonstante $\tau$ wurde bei einem Spannungswert von ca. $63 \%$ der Eingangsspannung gemessen und betr\"agt $230 \mu s$. Die Abweichung von $+1,70 \%$ kommt daher, dass es sich einerseits um reale Bauteile handelt und andererseits ein gewisser Messfehler vorliegt (Cusor am Oszilloskop zeigt nur runde Werte an).

\begin{figure}[H]
  \centering
  \includegraphics[width=100mm]{amplitudengang_rc_tiefpass.png}
  \caption{Amplitudengang RC-Tiefpass durch Messungen}
  \label{Figure04}
\end{figure}
\begin{figure}[H]
  \centering
  \includegraphics[width=100mm]{phasengang_rc_tiefpass.png}
  \caption{Phasengang RC-Tiefpass durch Messungen}
  \label{Figure05}
\end{figure}
\noindent In Abbildung~\ref{Figure04} und Abbildung~\ref{Figure05} sind Amplituden- und Phasengang des RC-Tiefpassfilters anhand der gemessenen Werte dargestellt. Die bieden Graphen entsprichen der erwarteten Form aus der Simulation. Bei der Grenzfrequenz ($723Hz$) wurde am Ausgang eine Spannung von $750mV$ und eine Phasenverschiebung von $-46^{\circ}$ gemessen. Diese Werte sind plausibel, da lt. Berechnung und Simulation die Ausgangsspannung $\frac{1V}{\sqrt{2}} \approx 707,11mV$ und die Phasenverschiebung $-45^{\circ}$ sind. Bis zur Frequenz von rund $10kHz$ hat der Amplitudengang, wie erwartet, eine Filtersteilheit von $-20db/Dekade$. Bei idealen Bauteilen w\"are der Spannungswert ab einer gewissen Frequenz $0 V$. Bei realen Bauteilen jedoch wird diese D\"ampfung ($0V$) nie erreicht, stattdessen geht die D\"ampfung, wie in Abbildung~\ref{Figure04} zu erkennen, asymptotisch gegen $0V$.




\section{Messung des Verhaltens eines RL-Filters 1. Ordnung}

\subsection{Aufgabenstellung}
Die charakteristischen Eigenschaften eines RL-Filters 1. Ordnung mit realen Bauteilen sollen untersucht werden und mit der Simulation verglichen werden.

\subsection{Schaltplan}
\begin{figure}[H]
\centering
\begin{circuitikz}[european]
  \draw
    (0,2) to [R, l=$R$, o-] (3,2);
  \draw
    (3,2) to [short, *-o] (4,2);
  \draw
    (3,2) to [L, l=$L$] (3,0);
  \draw
    (0,0) to [short, o-o] (4,0);
  \draw
    (0,2) to [open, v=$U_e$] (0,0);
  \draw
    (4,2) to [open, v^=$U_a$] (4,0);
\end{circuitikz}
\caption{RL Hochpassfilter 1.Ordnung}
\label{Figure06}
\end{figure}

\subsection{Durchf\"uhrung}
Die Schaltung wurde gem\"a\ss \, Schaltplan mit einem Widerstand $R=47\Omega$ und einem Kondensator $L=1mH$ aufgebaut. Um die Sprungantwort aufzuzeichnen wurde am Eingang ein periodisches Rechtecksignal mit $1 V_{PP}$, Offset $0,5 V$, $50 \%$ Duty Circle und Frequenz $2,5 kHz$, angelegt. Die Eingangs- und Ausgangsspannung wurden mit dem Oszilloskop im Zeitbereich aufgezeichnet (siehe Abbildung~\ref{Figure07}). Die Zeitkonstante wurde mit den gemessenen Bauteilwerten ($R=46,83 k\Omega$, $C=1,075mH$) berechnet und ergibt einen Wert von $\tau = \frac{L}{R} = 22,96 \mu s$. Die Zeitkonstante wurde aus der Sprungantwort ausgelesen indem der Zeitpunkt gemessen wurde, an dem das Signal der Sprungantwort rund $37 \%$ der Eingangsspannung betrug. \\
Am Eingang wurde nun ein Sinussignal mit $1 V_{PP}$, Offset $0 V$ angelegt. Die Eingangs- und Ausgangsspannung bzw. die Phasenverschiebung wurden mit dem Oszilloskop gemessen. Der Frequenzbereich von $100Hz$ bis $1MHz$ wurde mit 5 Frequenzmesspunkten pro Dekade aufgezeichnet. Weiters wurde ein Frequenzmesspunkt an der Grenzfrequenz gesetzt. Die Messpunkte wurden in tabellarischer Form aufgezeichnet und als Bodediagramm dargestellt (siehe Abbildung~\ref{Figure08} und Abbildung~\ref{Figure09}). Bei Frequenzen unter $100Hz$ konnten keine Vern\"unftigen Messungen der Ausgangsspannung bzw. der Phasenverschiebung durchgef\"uhrt werden, da das Signal am Oszilloskop zu schwar war.

\subsection{Ergebnis \& Diskussion}
\begin{figure}[H]
  \centering
  \includegraphics[width=100mm]{sprungantwort_rl_hochpass.png}
  \caption{Sprungantwort RL Hochpass}
  \label{Figure07}
\end{figure}
\noindent In der Abbildung~\ref{Figure02} ist zu sehen, wie sich die Ausgangsspannung bei Anlegen eines Rechtecksignals verh\"alt. Ein Sprung am Eingang bewirkt am Ausgang einen sprungartigen Anstieg der Spannung. Mit der Zeit geht die Ausgangsspannung asymptotisch gegen $0V$. Das l\"asst auf ein System mit Energiespeicher schlie\ss en. In diesem Fall handelt es sich um eine Spule, weil anfangs die Impedanz der Spule (und dadurch auch die Spannung) gro\ss \, ist, und mit der Zeit die Impedanz gegen $0 \Omega$ und somit auch die Ausgangsspannung gegen $0V$ geht. Bei der negativen Flanke des Rechtecksignals am Eingang, entsteht eine negative Spannung am Ausgang. Dieses Verhalten entsteht durch die gespeicherte Energie der Spule. \\
Weiters ist zu erkennen, dass sich die gemessene Eingangsspannung bei der positiven bzw. negativen Flanke \"ahnlich wie die Ausgangsspannung verh\"alt. Das ist auf die Reflexion der Spule im System zur\"uckzuf\"uhren. Die Spule wirkt also negativ auf den Frequenzgenerator aus. Um diesen Effekt zu vermeiden kann der Tiefpass auch mit einem RC-Glied realisiert werden.

\begin{table}[H]
\centering
\begin{tabular}{|c|c|c|c|}
\hline
& berechnet    & gemessen    & Abweichung \\ \hline
Zeitkonstante $\tau$ & $22,96 \mu s$  & $20 \mu s$      & $-2,96 \mu s (\approx-12,89 \%$) \\ \hline
\end{tabular}
\caption{Vergleich Zeitkonstante $\tau$ berechnet und bemessen}
\label{Figure03}
\end{table}

\noindent Die Zeitkonstante $\tau$ wurde bei einem Spannungswert von ca. $63 \%$ der Eingangsspannung gemessen und betr\"agt $20 \mu s$. Die Abweichung zum berechneten Wert von $-12,89 \%$ kommt einerseits daher, dass es sich einerseits um reale Bauteile handelt und ein gewisser Messfehler vorliegt (Cursor am Oszilloskop zeigt nur runde Werte).

\begin{figure}[H]
  \centering
  \includegraphics[width=100mm]{amplitudengang_rl_hochpass.png}
  \caption{Amplitudengang RL-Hochpass durch Messungen}
  \label{Figure08}
\end{figure}
\begin{figure}[H]
  \centering
  \includegraphics[width=100mm]{phasengang_rl_hochpass.png}
  \caption{Phasengang RL-Hochpass durch Messungen}
  \label{Figure09}
\end{figure}
\noindent In Abbildung~\ref{Figure07} sind Amplituden- und Phasengang des RL-Hochpassfilters anhand der gemessenen Werte dargestellt. Bei den Messungen wurde beobachtet, dass bei niedrigen Frequenzen die Eingangsspannung abf\"allt. Das kommt daher, dass das System durch die Spule viel Strom aufnimmt und der Funktionsgenerator den Strom nicht liefern kann. Trotzdem entsprichen die Graphen  der erwarteten Form aus der Simulation, da das Verh\"ahltnis $\frac{U_a}{U_e}$ gleich bleibt. Bis zur Grenzfrequnz ist, wie erwartet eine Filtersteilheit von rund $+20 db/Dekade$ zu erkennen, jedoch flacht die Kurve der bei niedriger Frequenz ab und geht asymptotisch gegen $0V$.\\
Bei der Grenzfrequenz ($7480 Hz$) wurde am Eingang $U_e = 650mV$, am Ausgang $U_a = 470mV$ und eine Phasenverschiebung von $-43^{\circ}$ gemessen. Das entspricht einer D\"ampfung von $\approx 72,31\%$. Die Werte sind plausibel, da bei idealen Bauelementen die D\"ampfung bei der Grenzfrequenz $-3db \approx 70,71\%$ und die Phasenverschiebung $-45^{\circ}$ ist.



\section{Messung des Verhaltens eines dynamischen Systems 2. Ordnung}

\subsection{Aufgabenstellung}
Die Sprungantwort sowie das Frequenzverhalten eines RLC-Systems 2. Ordnung soll untersucht werden.

\subsection{Schaltplan}
\begin{figure}[H]
  \centering
  \includegraphics[width=100mm]{rlc_schaltplan.png}
  \caption{RLC-System 2.Ordnung}
\end{figure}

\subsection{Durchf\"uhrung}
Die Schaltung wurde gem\"aß Schaltplan mit den Bauteilwerten $L=1mH$, $C=100nF$ und $R=22\Omega$ aufgebaut. Um die Sprungantwort des Systems mit dem Oszilloskop aufzuzeichnen, wurde als Eingangssignal eine periodische Rechteckschwingung mit $1Vpp$, Offset $0,5V$ und Frequenz $2,5kHz$ (Periodendauer $400\mu s$) angelegt. Weiters wurde das Frequenz- bzw. D\"ampfungsverhalten des Systems mit einem Sinussignal ($1Vpp$) ermittelt. Hierbei wurde zuerst die Resonanzfrequenz durch Variation der Frequenz des Eingangssignals bestimmt. Diese ist erreicht, wenn Eingangs- und Aussgangsignal eine Phasenverschiebung von $-90^{\circ}$ zueinander aufweisen. Um den Amplitudengang des Systems zu ermitteln, wurden Eingangs- und Ausgangspannung an verschiedenen Frequenzmesspunkten im lograithmischen Maßstab ermittelt. Im Bereich der Resonanzfrequenz wurden zusätzlich Messpunkte gewählt, um die Genauigkeit zu erhöhen. Die oben beschriebene Vorgangsweise wurde mit den Widerstandswerten $R = 183 \Omega$ (in der Angabe wurden $180\Omega$ verlangt, es gibt jedoch keinen Normwiderstand mit diesem Wert, darum wurden hier ein $150\Omega$ und ein $33\Omega$ in Serie geschalten) und $R = 1k\Omega$ wiederholt.

\subsection{Ergebnis \& Diskussion}
Die Resonanzfrequenz des Systems ergibt sich aus folgender Formel:
\begin{figure}[H]
  \centering
  $f_R = \sqrt{f_H*f_L} = \frac{1}{2\pi\sqrt{LC}} \approx 15916Hz$
\end{figure}

\begin{figure}[H]
  \centering
  \includegraphics[width=100mm]{sprungantwort_rlc_22.png}
  \caption{Sprungantwort bei $R=22\Omega$ (gelb $\hdots$ Eingangsspannung, gr\"un $\hdots$ Ausgangsspannung)}
\end{figure}
\noindent Durch den kleinen Widerstand ($R=22\Omega$) kommt es zu gut sichtbaren \"Uberschwingungen des $LC$-Glieds (Resonanzfall). Die Spule verursacht Schwingungen der Eingangsspannung.

\begin{figure}[H]
  \centering
  \includegraphics[width=150mm]{bode_rlc_22.png}
  \caption{Bode-Diagramm (Amplitudengang) bei $R=22\Omega$}
\end{figure}
\noindent Die gemessene Resonanzfrequenz betr\"agt hier $\sim15,8kHz$ (Abweichung $-0,7\%$). Von $10-2000Hz$ ist die Dämpfung $0dB$, das Ausgangssignal entspricht dem Eingangssignal. Im Bereich um $f_R$ erkennt man am Bode-Diagramm, dass hier tats\"achlich eine Verstärkung des Eingangsignals stattfindet. Dies lässt sich durch die kleine Dämpfung bei $R=22\Omega$ erkl\"aren, wodurch sich das $LC$-Glied im Resonanzfall befindet, es kommt zu \"Uberschwingungen. Nach dem die Resonanzfrequenz erreicht wurde ist die D\"ampfung sehr stark, danach l\"asst sich die typische Filtersteilheilt eines Tiefpassfilters 2.Ordnung von $-40dB/Dekade$ gut erkennen.

\begin{figure}[H]
  \centering
  \includegraphics[width=150mm]{phasengang_rlc_22.png}
  \caption{Phasengang bei $R=22\Omega$}
\end{figure}

\noindent Bei $-90^{\circ}$ Phasenverschiebung wird die Resonanzfrequenz erreicht.

\begin{figure}[H]
  \centering
  \includegraphics[width=100mm]{sprungantwort_rlc_180.png}
  \caption{Sprungantwort bei $R=183\Omega$ (gelb $\hdots$ Eingangsspannung, gr\"un $\hdots$ Ausgangsspannung)}
\end{figure}
\noindent Bei $R=183\Omega$ sind gerade keine \"Uberschwingungen sichtbar (aperiodischer Grenzfall). Die Sprungantwort zeigt das typische Tiefpassverhalten. Die Schwingung an der Eingangsspannung wird durch die Induktivit\"at verursacht.

\begin{figure}[H]
  \centering
  \includegraphics[width=150mm]{bode_rlc_180.png}
  \caption{Bode-Diagramm (Amplitudengang) bei $R=183\Omega$}
\end{figure}

\noindent Die gemessene Resonanzfrequenz betr\"agt hier $\sim15,5kHz$ (Abweichung $-2,61\%$). Das System befindet sich im aperiodischen Grenzfall. Von $10-2000Hz$ ist die Dämpfung $0dB$, das Ausgangssignal entspricht dem Eingangssignal. Auch hier lässt sich die typische Filtersteilheit von $-40dB/Dekade$ nach dem Erreichen der Resonanzfrequenz aus dem Diagramm ablesen.

\begin{figure}[H]
  \centering
  \includegraphics[width=150mm]{phasengang_rlc_180.png}
  \caption{Phasengang bei $R=183\Omega$}
\end{figure}

\noindent Bei $-90{\circ}$ Phasenverschiebung wird die Resonanzfrequenz erreicht.

\begin{figure}[H]
  \centering
  \includegraphics[width=80mm]{sprungantwort_rlc_1k.png}
  \caption{Sprungantwort bei $R=1k\Omega$ (gelb $\hdots$ Eingangsspannung, gr\"un $\hdots$ Ausgangsspannung)}
\end{figure}
\noindent Durch den großen Widerstand von $R=1k\Omega$ ist die Zeitkonstante des Systems sehr groß. Dadurch wird die am Eingang anliegende Rechteckspannung am Ausgang stark verschliffen.

\begin{figure}[H]
  \centering
  \includegraphics[width=150mm]{bode_rlc_1k.png}
  \caption{Bode-Diagramm (Amplitudengang) bei $R=1k\Omega$}
\end{figure}

\noindent Die Grenzfrequenz des Kondensators liegt bei:
\begin{figure}[H]
  \centering
  $f_C = \frac{1}{2\pi RC} = \frac{1}{2\pi *1k\Omega*100nF} = 1,59kHz$
\end{figure}

\noindent Die Grenzfrequenz der Spule liegt bei:
\begin{figure}[H]
  \centering
$f_L = \frac{R}{2\pi L} = \frac{1k\Omega}{2\pi *1mH} \approx 159,15kHz$
\end{figure}

\noindent Durch den großen Widerstand wird die Grenzfrequenz des Kondensators $f_C$ (orange) früher erreicht, sie liegt laut Messung bei $\sim1,6kHz$. Die Grenzfrequenz der Spule $f_L$ (gr\"un) wird laut Messung erst bei $\sim145kHz$ erreicht. Der Filter beginnt hier sehr früh zu Dämpfen, jedoch wird im Bereich zwischen $f_C$ und $f_L$ nur eine Filtersteilheit von $-20dB/Dekade$ erreicht.

\begin{figure}[H]
  \centering
  \includegraphics[width=150mm]{phasengang_rlc_1k.png}
  \caption{Phasengang bei $R=1k\Omega$}
\end{figure}

\noindent Hier l\"asst sich gut erkennen, dass die Grenzfrequenz des Kondensators fr\"uher erreicht wird als jene der Spule, die Änderung der Phasenverschiebung ist im Bereich der beiden Grenzfrequenzen gr\"oßer.\\\\

\noindent Der Widerstand hat aus rein mathematischer Sicht keinen Einfluss auf die Resonanzfrequenz, wie man an obenstehender Formel sehen kann. Bei den Messungen zeigen sich jedoch leichte Abweichungen von der berechneten Resonanzfrequenz, diese sind auf die Eigenschaften realer Bauteile zurückzuführen.\\\\

\noindent Im Vergleich zu den Simulationen lässt sich sagen, dass das simulierte Verhalten des Filters mit verschiedenen Widerständen im Labor sehr gut messbar und nachvollziehbar war. Es zeigen sich jedoch einige Ungenauigkeiten, die auf reale Bauteile und Messfehler zurückzuführen sind.\\\\

\noindent Die \"Ubertragungsfunktion $H(s)$ des Systems lautet:
\begin{figure}[H]
  \centering
  $\frac{U_a}{U_e} = \frac{Z_C}{Z_R+Z_L+Z_C} = \frac{\frac{1}{j\omega C}}{R+j\omega L+\frac{1}{j\omega C}} = \frac{1}{j\omega CR-\omega^2CL+1} \Rightarrow H(s) = \frac{1}{s^2CL+sCR+1}$
\end{figure}


\begin{figure}[H]
  \centering
  \includegraphics[width=150mm]{pnd.jpg}
  \caption{PN-Diagramm der \"Ubertragungsfunktion (violett: $R=22\Omega$, blau: $R=180\Omega$, gr\"un: $R=1k\Omega$)}
\end{figure}

\noindent Bei $R=22\Omega$ und $R=180\Omega$ hat das System jeweils zwei konjugiert komplexe Polstellen, bei $R=1k\Omega$ zwei reelle Polstellen. Da alle Punkte in der linken Halbebene liegen, ist das System stabil.

\section{Anhang - Messwerte}

\begin{figure}[H]
  \centering
  \begin{minipage}[b]{0.4\textwidth}
    \includegraphics[width=\textwidth]{rc_data.png}
    \caption{Messdaten RC-Tiefpass}
  \end{minipage}
  \hfill
  \begin{minipage}[b]{0.4\textwidth}
    \includegraphics[width=\textwidth]{rl_data.png}
    \caption{Messdaten RL-Hochpass}
  \end{minipage}
\end{figure}

\begin{figure}[H]
  \centering
  \begin{minipage}[b]{0.3\textwidth}
    \includegraphics[width=\textwidth]{rlc_22_data.png}
    \caption{Messdaten RLC-Tiefpass mit $R=22\Omega$}
  \end{minipage}
  \hfill
  \begin{minipage}[b]{0.3\textwidth}
    \includegraphics[width=\textwidth]{rlc_180_data.png}
    \caption{Messdaten RLC-Tiefpass mit $R=183\Omega$}
  \end{minipage}
  \hfill
  \begin{minipage}[b]{0.3\textwidth}
    \includegraphics[width=\textwidth]{rlc_1k_data.png}
    \caption{Messdaten RLC-Tiefpass mit $R=1k\Omega$}
  \end{minipage}
\end{figure}

\end{document}
