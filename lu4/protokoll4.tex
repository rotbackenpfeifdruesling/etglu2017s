% -*- TeX:de -*-
\NeedsTeXFormat{LaTeX2e}
\documentclass[12pt,a4paper,titlepage]{article}

%\usepackage[german]{babel} % german text
\usepackage[DIV12]{typearea} % size of printable area
\usepackage[T1]{fontenc} % font encoding
\usepackage[utf8]{inputenc} % probably on Linux

\usepackage{graphicx} % to include images
\graphicspath{ {img/} } % set default image directory
\usepackage{subfigure} % for creating subfigures
\usepackage{amsmath} % a bunch of symbols
\usepackage{amssymb} % even more symbols
\usepackage{booktabs} % pretty tables
\usepackage{csquotes}

% a floating environment for circuits
\usepackage{float}
\usepackage{caption}

\newfloat{circuit}{tbph}{circuits}
\floatname{circuit}{Schaltplan}

% a floating environment for diagrams
\newfloat{diagram}{tbph}{diagrams}
\floatname{diagram}{Diagramm}

\renewcommand{\familydefault}{\sfdefault} % activate to use sans-serif font as default

\sloppy % friendly typesetting

\usepackage{eurosym}
\usepackage{makeidx}
\usepackage{amsfonts}
\usepackage{mparhack}
\usepackage{array}
\usepackage{tabularx}
\usepackage{minitoc}
\usepackage[colorlinks=true]{hyperref}
\usepackage{epstopdf}
\usepackage{setspace}
\usepackage{csquotes}
\usepackage{circuitikz}

% hyperref settings
\hypersetup{
    colorlinks=false,       % false: boxed links; true: colored links
    linkcolor=black,          % color of internal links (change box color with linkbordercolor)
    citecolor=black,        % color of links to bibliography
    filecolor=black,      % color of file links
    urlcolor=black           % color of external links
}

\begin{document}

\begin{titlepage}

\begin{figure*}[h!]
  \includegraphics[width=8cm]{TULogo_CMYK}
\end{figure*}

\begin{center}
\vspace*{1.3cm}
{\Huge Elektrotechnische Grundlagen der Informatik\\(LU 182.692)\\}
\vspace{1.7cm}
{\LARGE Protokoll der 4. Laborübung: \enquote{Spektren}\\}
\vspace{1.7cm}

% fill in group number and date of lab here
% CHANGE ME!
{\Large Gruppennr.: 22 \hspace{1cm} Datum der Labor\"ubung: 21.06.2017}

% fill in IDs and names here
% CHANGE ME!
\begin{table}[h!]
\centering
\begin{tabular}{|p{3.5cm}|p{3.5cm}|p{6.5cm}|}
\hline \textbf{Matr. Nr.} & \textbf{Kennzahl} & \textbf{Name} \\
\hline
1614835 & 033 535 & Jan Nausner \\
\hline
1633068 & 033 535 & David Pernerstorfer \\
\hline
\end{tabular}
\end{table}

\end{center}
\vspace{1.0cm}

\begin{table}[h!]
\begin{tabular}{|l|l|}
\hline \textbf{Kontrolle} & \checkmark \\
\hline Sinus-Signal im Frequenzbereich & \\
\hline Rechteck-Signal im Frequenzbereich & \\
\hline Amplitudenmodulation & \\
\hline Brückengleichrichter & \\
\hline
\end{tabular}
\end{table}

\end{titlepage}
% start of actual lab protocol
% CHANGE ME!

\setcounter{page}{2}

\newpage
\setcounter{tocdepth}{1}
\tableofcontents

\newpage

\section*{Materialien}
\begin{itemize}
	\item Oszilloskop: Agilent InfiniiVision MSO-X 3054A
	\item Frequenzgenerator: Agilent 33220A
  \item Netzteil Agilent U8031A
  \item Multimeter: Amprobe 37XR-A
\end{itemize}

\section{Messung eines Sinussignals im Spektralbereich mittels FFT}

\subsection*{Notizen}
Eingangssignal: Sinus $100kHz$, $1V_{pp}$, $V_{offset}=0V$ \\
Aufzeichnung Signal im Zeitbereich scope\_18/19, kein Aliasing \\
\\
Aufzeichung im FQ bereich: Einstellungen: Taste Math;

logarithmisch (spanne 200kHz, center 100kHz):
Hanning-Fenster: scope_20/21
Rechteck-Fenster: scope_22/23
BLackmann-Harris: scope_24/25
Rechteck: scope_30/31 spanne 10kHz, center 100kHz
Rechteck: scope_34 spanne 10kHz, center 100kHz Messung
Blackman-Harris: scope_35 spanne 10kHz, center 100kHz Messung
Blackman-Harris: scope_36 spanne 10kHz, center 100kHz Messung, Amplitude

linear:
Rechteck: scope_26/27 spanne 1kHz, center 100kHz
Rechteck: scope_28/29 spanne 10kHz, center 100kHz
Blackman-Harris: scope_32/ spanne 10kHz, center 100kHz Messung
Rechteck: scope_33 spanne 10kHz, center 100kHz Messung
Blackman-Harris: scope_37 spanne 10kHz, center 100kHz Messung, Amplitude

Messung: Amplitude: -9.2dB = 20*log_10(Veff/1V) = 20*log_10(0.5*1/sqrt(2))


\subsection*{Aufgabenstellung}

\subsection*{Schaltplan}
% \begin{figure}[H]
%   \centering
%   \includegraphics[width=100mm]{ni_opv_schaltung.png}
%   \caption{Nichtinvertierender OPV}
% \end{figure}

\subsection*{Durchf\"uhrung}

\subsection*{Ergebnisse \& Diskussion}




\section{Messung eines Rechtecksignals}

\subsection*{Notizen}
FQ Generator Einstellungen: Square $50\%$ duty circle $10kHz$, $1V_{pp}$ \\
Rechtecksignal im Zeitbereich: scope_40 \\
Hanning; Vrms; spanne 100kHz, center 50kHz; Vrm $450mV$; scope_39 \\
Hauptkomponente gemessen mit Cursor: $10kHz$, $450mV$ \\
1. Nebenkomponente: $30kHz$, $147mV$
2. Nebenkomponente: $50kHz$, $94mV$
3. Nebenkomponente: $70kHz$, $65mV$



\subsection*{Schaltplan}
% \begin{figure}[H]
%   \centering
%   \includegraphics[width=100mm]{i_opv_schaltung.png}
%   \caption{Invertierender OPV}
% \end{figure}

\subsection*{Durchf\"uhrung}

\subsection*{Ergebnisse \& Diskussion}



\section{Amplitudenmodulation}

\subsection*{Notizen}
Einstellungen FQ Generator: \\

Tr\"agerfq: Sinus $100kHz$, $1V_{pp}$ \\
Nutzsignal: Sinus $1kHz$, Gleichanteil $0V$ \\
Zeitbereich scope_41 \\
Zeitbereich scope_42 \\
Hanning-Fenster am Oszi

Frequenzbereich: scope_43, linear
scope_47 logarithmisch
scope_48 logarithmisch (rectangle)
Hauptkomponente: 100kHz, Amplitude 166,92mV
Linke Nebenkomponente: 99kHz, 86mV
Rechte Nebenkomponente: 101kHz, 86mV

Einstellungen FQ Generator: \\
Tr\"agerfq: Sinus $100kHz$, $1V_{pp}$ \\
Nutzsignal: Rechteck $1kHz$, Gleichanteil $0V$ \\
Zeitbereich scope_44 \\

Frequenzbereich: scope_45, linear
scope_46 logarithmisch
Hauptkomponente: 100kHz, Amplitude 175mV
1. Linke Nebenkomponente: 99kHz, 111mV
2. Linke Nebenkomponente: 97kHz, 37mV
3. Linke Nebenkomponente: 95kHz, 23mV
4. Linke Nebenkomponente: 93kHz, 16mV
1. R Nebenkomponente: 101kHz, 111mV
2. R Nebenkomponente: 103kHz, 37mV
3. R Nebenkomponente: 105kHz, 23mV
4. R Nebenkomponente: 107kHz, 16mV


\subsection*{Aufgabenstellung}

\subsection*{Schaltplan}
% \begin{figure}[H]
%   \centering
%   \includegraphics[width=80mm]{integrierer_opv_schaltung.png}
%   \caption{Integrierer}
%   \label{figure31}
% \end{figure}

\subsection*{Durchf\"uhrung}

\subsection*{Ergebnisse \& Diskussion}



\section{Brückengleichrichter}

\subsection*{Notizen}
Dioden N4148, Widerstand $1MOhm$ \\

Frequenzgenerator $2V_{pp}$, $1kHz$
Zeitbereich: ($U_a$) $scope_52$, $scope_53$ \\
Messung $U_a$ $520mV_PP$ $DC_{rms}=338mV$ $Fq=2kHz$ \\
\\
Frequenzgenerator $10V_{pp}$, $1kHz$
Zeitbereich: ($U_a$) $scope_54$ \\
Messung $U_a$ $4,22V_PP$ $DC_{rms}=2,91V$ $Fq=2kHz$ \\
\\


Frequenzbereich:
Spektrum scope_57 (rectangle), logarithmisch \\
Hauptkomponente: $2kHz$, $3,13dB$ \\
1. Nebenkomponente: $4kHz$, $-11,27dB$ \\
2. Nebenkomponente: $6kHz$, $-21,9dB$ \\
3. Nebenkomponente: $8kHz$, $-27,86dB$ \\
4. Nebenkomponente: $10kHz$, $-37,25dB$


\subsection*{Aufgabenstellung}

\subsection*{Schaltplan}
% \begin{figure}[H]
%   \centering
%   \includegraphics[width=80mm]{i_schmitt_trigger_schaltung.png}
%   \caption{Invertierender Schmitt-Trigger}
%   \label{figure41}
% \end{figure}


\subsection*{Durchf\"uhrung}

\newpage

\noindent Berechnung der Fourierreihe des gleichgerichteten Sinus:\\
\begin{align*}
    a_n &= \frac{2}{\pi}\int_{0}^{\pi}sin(t)\cdot cos(nt) dt\\
    &= \frac{1}{\pi}\int_{0}^{\pi}\left[sin(t - nt) + sin(t + nt)\right] dt\\
    &= \frac{1}{\pi}\left[\frac{1}{n-1}cos(t(1-n)) - \frac{1}{1+n}cos(t(1+n))\right]\bigr\rvert_{0}^{\pi}\\
    &= \frac{1}{\pi}\left[\frac{1}{n-1}cos(\pi-n\pi) - \frac{1}{1+n}cos(n\pi + \pi) - \frac{1}{n-1} + \frac{1}{1+n}\right]\\
    &= \frac{2cos(n\pi + \pi) - 2}{\pi(n+1)(n-1)} = \left\{
	    \begin{array}{ll}
		     0  & n \; ungerade \\
		     \frac{-4}{\pi(n+1)(n-1)} & n \; gerade
	    \end{array}
    \right.\\
    a_0 &= \frac{-4}{\pi(0+1)(0-1)} = \frac{4}{\pi}\\
    |sin(\omega t)| &= \frac{2}{\pi} - \sum_{n=1}^{\infty} \frac{4 \cdot cos(2n\omega t)}{\pi(2n+1)(2n-1)}
\end{align*}

\subsection*{Ergebnisse \& Diskussion}

\section{Anhang - Messwerte}

% \begin{figure}[H]
%   \centering
%   \begin{minipage}[b]{0.4\textwidth}
%     \includegraphics[width=\textwidth]{daten_471.png}
%     \caption{Messdaten invertierender OPV (-47 Verstärkung)}
%   \end{minipage}
%   \hfill
%   \begin{minipage}[b]{0.4\textwidth}
%     \includegraphics[width=\textwidth]{daten_4_7.png}
%     \caption{Messdaten invertierender OPV (-4,7 Verstärkung)}
%   \end{minipage}
% \end{figure}

\end{document}
