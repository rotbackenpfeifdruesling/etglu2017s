% -*- TeX:de -*-
\NeedsTeXFormat{LaTeX2e}
\documentclass[12pt,a4paper,titlepage]{article}

%\usepackage[german]{babel} % german text
\usepackage[DIV12]{typearea} % size of printable area
\usepackage[T1]{fontenc} % font encoding
\usepackage[utf8]{inputenc} % probably on Linux

\usepackage{graphicx} % to include images
\graphicspath{ {img/} } % set default image directory
\usepackage{subfigure} % for creating subfigures
\usepackage{amsmath} % a bunch of symbols
\usepackage{amssymb} % even more symbols
\usepackage{booktabs} % pretty tables
\usepackage{csquotes}

% a floating environment for circuits
\usepackage{float}
\usepackage{caption}

\newfloat{circuit}{tbph}{circuits}
\floatname{circuit}{Schaltplan}

% a floating environment for diagrams
\newfloat{diagram}{tbph}{diagrams}
\floatname{diagram}{Diagramm}

\renewcommand{\familydefault}{\sfdefault} % activate to use sans-serif font as default

\sloppy % friendly typesetting

\usepackage{eurosym}
\usepackage{makeidx}
\usepackage{amsfonts}
\usepackage{mparhack}
\usepackage{array}
\usepackage{tabularx}
\usepackage{minitoc}
\usepackage[colorlinks=true]{hyperref}
\usepackage{epstopdf}
\usepackage{setspace}
\usepackage{csquotes}

% hyperref settings
\hypersetup{
    colorlinks=false,       % false: boxed links; true: colored links
    linkcolor=black,          % color of internal links (change box color with linkbordercolor)
    citecolor=black,        % color of links to bibliography
    filecolor=black,      % color of file links
    urlcolor=black           % color of external links
}

\begin{document}

\begin{titlepage}

\begin{figure*}[h!]
  \includegraphics[width=8cm]{TULogo_CMYK}
\end{figure*}

\begin{center}
\vspace*{1.3cm}
{\Huge Elektrotechnische Grundlagen der Informatik\\(LU 182.692)\\}
\vspace{1.7cm}
{\LARGE Protokoll der 2. Labor\"ubung: \enquote{Filter}\\}
{\large \enquote{Transiente Vorg\"ange und Frequenzverhalten}\\}
{\LARGE b) Messungen\\}
\vspace{1.5cm}

% fill in group number and date of lab here
% CHANGE ME!
{\Large Gruppennr.: 22 \hspace{1cm} Datum der Labor\"ubung: 19.05.2017}

% fill in IDs and names here
% CHANGE ME!
\begin{table}[h!]
\centering
\begin{tabular}{|p{3.5cm}|p{3.5cm}|p{6.5cm}|}
\hline \textbf{Matr. Nr.} & \textbf{Kennzahl} & \textbf{Name} \\
\hline
1614835 & 033 535 & Jan Nausner \\
\hline
1633068 & 033 535 & David Pernerstorfer \\
\hline
\end{tabular}
\end{table}

\end{center}
\vspace{1.0cm}

\begin{table}[h!]
\begin{tabular}{|l|l|}
\hline \textbf{Kontrolle} & \checkmark \\
\hline Verhalten eines Filters 1. Ordnung & \\
\hline Verhalten eines RL-Filters & \\
\hline Dynamisches System 2. Ordnung & \\
\hline
\end{tabular}
\end{table}

\end{titlepage}
% start of actual lab protocol
% CHANGE ME!

\setcounter{page}{2}

\newpage
\setcounter{tocdepth}{1}
\tableofcontents

\newpage

\section*{Materialien}
\begin{itemize}
	\item Oszilloskop:
	\item Frequenzgenerator:
\end{itemize}

\section{Messung des Verhaltens eines RC-Filters 1. Ordnung}

\subsection{Aufgabenstellung}

\subsection{Schaltplan}

\subsection{Durchf\"uhrung}

\subsection{Ergebnis \& Diskussion}

\section{Messung des Verhaltens eines RL-Filters 1. Ordnung}

\subsection{Aufgabenstellung}

\subsection{Schaltplan}

\subsection{Durchf\"uhrung}

\subsection{Ergebnis \& Diskussion}

\section{Messung des Verhaltens eines dynamischen Systems 2. Ordnung}

\subsection{Aufgabenstellung}

\subsection{Schaltplan}

\subsection{Durchf\"uhrung}

\subsection{Ergebnis \& Diskussion}

\end{document}
