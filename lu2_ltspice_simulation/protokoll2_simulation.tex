% -*- TeX:de -*-
\NeedsTeXFormat{LaTeX2e}
\documentclass[12pt,a4paper,titlepage]{article}

%\usepackage[german]{babel} % german text
\usepackage[DIV12]{typearea} % size of printable area
\usepackage[T1]{fontenc} % font encoding
\usepackage[utf8]{inputenc} % probably on Linux

\usepackage{graphicx} % to include images
\graphicspath{ {img/} } % set default image directory
\usepackage{subfigure} % for creating subfigures
\usepackage{amsmath} % a bunch of symbols
\usepackage{amssymb} % even more symbols
\usepackage{booktabs} % pretty tables
\usepackage{csquotes}

% a floating environment for circuits
\usepackage{float}
\usepackage{caption}

\newfloat{circuit}{tbph}{circuits}
\floatname{circuit}{Schaltplan}

% a floating environment for diagrams
\newfloat{diagram}{tbph}{diagrams}
\floatname{diagram}{Diagramm}

\renewcommand{\familydefault}{\sfdefault} % activate to use sans-serif font as default

\sloppy % friendly typesetting

\usepackage{eurosym}
\usepackage{makeidx}
\usepackage{amsfonts}
\usepackage{mparhack}
\usepackage{array}
\usepackage{tabularx}
\usepackage{minitoc}
\usepackage[colorlinks=true]{hyperref}
\usepackage{epstopdf}
\usepackage{setspace}
\usepackage{csquotes}

\begin{document}

\begin{titlepage}

\begin{figure*}[h!]
  \includegraphics[width=8cm]{TULogo_CMYK}
\end{figure*}

\begin{center}
\vspace*{1.3cm}
{\Huge Elektrotechnische Grundlagen der Informatik\\(LU 182.692)\\}
\vspace{1.7cm}
{\LARGE Protokoll der 2. Laborübung: \enquote{Filter}\\}
{\large \enquote{Transiente Vorgänge und Frequenzverhalten}\\}
{\LARGE a) LTSPICE-Simulationen\\}
\vspace{1.5cm}

% fill in group number and date of lab here
% CHANGE ME!
{\Large Gruppennr.: \ldots \hspace{1cm} Datum der Laborübung: \ldots\ldots\ldots\ldots\ldots}

% fill in IDs and names here
% CHANGE ME!
\begin{table}[h!]
\centering
\begin{tabular}{|p{3.5cm}|p{3.5cm}|p{6.5cm}|}
\hline \textbf{Matr. Nr.} & \textbf{Kennzahl} & \textbf{Name} \\
\hline
1614835 & & Jan Nausner \\
\hline
1633068 & & David Pernerstorfer \\
\hline
\end{tabular}
\end{table}

\end{center}
\vspace{1.0cm}

\begin{table}[h!]
\begin{tabular}{|l|l|}
\hline \textbf{Kontrolle} & \checkmark \\
\hline Verhalten eines Filters 1. Ordnung & \\
\hline Verhalten eines RL-Filters & \\
\hline Dynamisches System 2. Ordnung & \\
\hline
\end{tabular}
\end{table}

\end{titlepage}
\setcounter{page}{2}

% start of actual lab protocol
% CHANGE ME!
\newpage
\section{RC-Tiefpassilter 1. Ordnung}

\subsection{Aufgabenstellung}
Die Sprungantwort und der Amplituden- bzw. Phasengang eines RC-Tiefpassilters 1. Ordnung soll mittels LTSpice simuliert werden.

\subsection{Schaltplan}

\begin{figure}[H]
  \centering
  \includegraphics[width=100mm]{filter01_schaltung.PNG}
  \caption{RC-Tiefpassfilter 1. Ordnung}
\end{figure}

\subsection{Durchführung}
Zuerst wird die RC-Schaltung mit $R1 = 22k\Omega$ und $C1 = 10nF$ zusammengefügt. Dazu wird mit der PULSE-Option der Sprung von $0V$ auf $1V$ angelegt. Die Sprungantwort des Systems wird dann im Bereich von $0s$ bis $2ms$ geplotted. Nun soll der Amplituden- und Phasengang simuliert werden. Dazu wird eine sinusförmige Spannung mit $1V_{pp}$ ($0,5V$ Amplitude) angelegt. Es wird eine dekadische Simulation im Bereich von $1Hz$ bis $1MHz$ durchgeführt und das daraus resultierende Bode-Diagramm aufgezeichnet.

\subsection{Ergebnis}

\begin{figure}[H]
  \centering
  \includegraphics[width=150mm]{sprungantwort_filter01.png}
  \caption{Sprungantwort RC-Tiefpassfilter 1. Ordnung}
\end{figure}

Anhand dieses Diagramms erkennt man sehr gut, wie der Strom $I(C1)$ am Kondensator zuerst maximal ist und bei steigender Spannung $V(n002)$ am Kondensator immer stärker abfällt. Der Strom erreicht seinen Tiefpunkt, sobald die Spannung am Kondensator maximal ist.\\\\
Die Zeitkonstante
\begin{figure}[H]
  \centering
  $\tau = R1*C1 = 22k\Omega*10nF = 220\mu s$
\end{figure}
\noindent besagt, dass nach $222\mu s$ 63\% der Maximalspannung von $C1$ erreicht wird. Dies lässt sich gut am Diagramm ablesen, wo die Spannung $V(n002)$ bei $0,22ms$ ca. $0,63V$ beträgt. Nach $5*\tau = 1,1ms$ wird 99\% der Maximalspannung erreicht, der Kondensator ist de facto vollständig geladen, die Spannung ist maximal.

\begin{figure}[H]
  \centering
  \includegraphics[width=150mm]{bode_filter01.png}
  \caption{Bode-Diagramm RC-Tiefpassfilter 1. Ordnung}
\end{figure}

\noindent Ein Bode-Diagramm dient der Darstellung der Filtereigenschaften. Auf der X-Achse wird hierfür die Frequenz in dekadischen Abständen aufgetragen und auf der Y-Achse das Übertragungsverhalte $20\log(\frac{U_a}{U_e})$ bzw. der Phasengang.

\noindent Das Übertragungsverhalten, welches als Quotient von Ausgangs- und Eingangsspannung definiert ist, ergibt sich aus folgender Formel:
\begin{figure}[H]
  \centering
  $\frac{U_a}{U_e} = \frac{Z_{C1}}{Z_{R1}+Z_{C1}} = \frac{\frac{1}{j\omega C1}}{R1+\frac{1}{j\omega C1}} = \frac{1}{1+j\omega R1C1}$
\end{figure}
Wenn die Frequenz bei 0 liegt, dann sind Aus- und Eingangsspannung gleich groß. Geht die Frequenz gegen $\infty$ lässt der Filter kein Signal mehr durch.

\noindent Die Grenzfrequenz des Filters wird bei
\begin{figure}[H]
  \centering
  $f_c = \frac{1}{2\pi R1C1} \approx 723Hz$
\end{figure}
\noindent erreicht.\\
Hier beträgt die Dämpfung ca. $-3dB$ und der Phasengang $-45^{\circ}$ und die Spannung über $R1$ und $C1$ ist gleich. Man kann anhand des Diagramms auch gut erkennen, dass die Filtersteilheit bei einem RC-Tiefpass 1.Ordnung $-20dB/Dekade$ beträgt.


\newpage
\section{RL-Hochpassfilter 1. Ordnung}

\subsection{Aufgabenstellung}
Die Sprungantwort und der Amplituden- bzw. Phasengang eines RL-Hochpassfilters 1. Ordnung soll mittels LTSpice simuliert werden.

\subsection{Schaltplan}

\begin{figure}[H]
  \centering
  \includegraphics[width=100mm]{filter02_schaltung.PNG}
  \caption{RL-Hochpassfilter 1. Ordnung}
\end{figure}

\subsection{Durchführung}
Zuerst wird die RL-Schaltung mit $R1 = 47\Omega$, $L1 = 1mH$ und $R_L = 0 \Omega$ zusammengefügt. Dazu wird mit der PULSE-Option der Sprung von $0V$ auf $1V$ angelegt. Die Sprungantwort des Systems wird dann im Bereich von $0s$ bis $200\mu s$ geplotted. Nun soll der Amplituden- und Phasengang simuliert werden. Dazu wird eine sinusförmige Spannung mit $1V_{pp}$ ($0,5V$ Amplitude) angelegt. Es wird eine dekadische Transientensimulation im Bereich von $1Hz$ bis $1MHz$ durchgeführt und das daraus resultierende Bode-Diagramm aufgezeichnet. Danach wird der Widerstand $R_L$ auf $1,2\Omega$ geändert, die Simulation des Amplituden- und Phasenganges wiederholt und das neue Bode-Diagramm aufgezeichnet.

\subsection{Ergebnis}

\begin{figure}[H]
  \centering
  \includegraphics[width=150mm]{sprungantwort_filter02.png}
  \caption{Sprungantwort RL-Hochpassfilter 1. Ordnung}
\end{figure}

Anhand dieses Diagramms erkennt man sehr gut, wie der Strom $I(L1)$ an der Spule zuerst minimal ist und bei fallender Spannung $V(n002)$ an der Spule immer stärker ansteigt. Der Strom erreicht sein Maximum, sobald die Spannung an der Spule minimal ist.\\\\
Die Zeitkonstante
\begin{figure}[H]
  \centering
  $\tau = \frac{L1}{R1} = \frac{1mH}{47} \approx 21,3\mu s$
\end{figure}
\noindent besagt, dass $L1$ nach $21,3\mu s$ zu 63\% aufmagnetisiert ist. Nach $5*\tau \approx 106,4\mu s$ ist $L1$ zu 99\% aufmagnetisiert, der Strom is maximal.

\begin{figure}[H]
  \centering
  \includegraphics[width=150mm]{bode_filter02.png}
  \caption{Bode-Diagramm RL-Hochpassfilter 1. Ordnung}
\end{figure}

\noindent Das Übertragungsverhalten, welches als Quotient von Ausgangs- und Eingangsspannung definiert ist, ergibt sich aus folgender Formel:
\begin{figure}[H]
  \centering
  $\frac{U_a}{U_e} = \frac{Z_{L1}}{Z_{R1}+Z_{L1}} = \frac{j\omega L1}{R1 + j\omega L1}$
\end{figure}
Wenn die Frequenz bei 0 liegt, dann dann lässt der Filter kein Signal durch. Geht die Frequenz gegen $\infty$, dann sind Aus- und Eingangsspannung gleich groß.

\noindent Die Grenzfrequenz des Filters wird bei
\begin{figure}[H]
  \centering
  $f_c = \frac{R1}{2*\pi*L1} \approx 7,5kHz$
\end{figure}
\noindent erreicht.\\
Hier beträgt die Dämpfung ca. $3dB$ und der Phasengang $45^{\circ}$ und die Spannung über $R1$ und $L1$ ist gleich. Man kann anhand des Diagramms auch gut erkennen, dass die Filtersteilheit bei einem RL-Tiefpass 1.Ordnung $20dB/Dekade$ beträgt.

\begin{figure}[H]
  \centering
  \includegraphics[width=150mm]{bode_filter02R.png}
  \caption{Bode-Diagramm RL-Hochpassfilter 1. Ordnung mit Serienwiderstand $1,2\Omega$}
\end{figure}

\noindent Durch den parasitären Serienwiderstand wird die Schaltung verändert und somit auch die Filtereigenschaften. Verglichen mit dem Bode-Diagramm für das System ohne parasitären Serienwiderstand zeigt sich, dass hier eine starke Veränderung des gewünschten Hochpassverhaltens eintritt und dieses kaum mehr vorhanden ist.


\newpage
\section{Simulation eines dynamischen Systems 2. Ordnung}
\subsection{Aufgabenstellung}
Der Amplituden- und Phasengang eines dynamischen Systems 2. Ordnung soll mittels LT-Spice simuliert werden.

\subsection{Schaltplan}
\begin{figure}[H]
  \centering
  \includegraphics[width=100mm]{filter03_schaltung.PNG}
  \caption{RLC-System}
  \label{aufgabe3_schaltplan}
\end{figure}

\subsection{Durchf\"uhrung}
Zuerst wird die Schaltung mit $R = 22 \Omega$, $L = 1 mH$ ($Rs = 0 \Omega$) und $C = 100 nF$ zusammengef\"ugt. Der Frequenzbereich von $1 Hz$ bis $1 MHz$ wird simuliert. Die entsprechende Einstellung findet man unter \textit{Simulate} - \textit{Edit Simulation Command} - Reiter \textit{AC Analysis}. Folgende Einstellungen werden vorgenommen.
\begin{itemize}
  \item Type of Sweep: \textit{Decade}
  \item Number of Points per Decade: 100
  \item Start frequency: 1
  \item Stop frequency: 1MEG
\end{itemize}
\noindent Nun wird die Simulation gestartet und Phasengang und Amplitudengang werden geplottet. Die gleiche Simulation wird mit ge\"anderten Widerstandswerten durchgef\"uhrt ($180 \Omega$, $1 k\Omega$).

\subsection{Ergebnis}
\begin{figure}[H]
  \centering
  \includegraphics[width=150mm]{bode_filter03_22o.PNG}
  \caption{Bode-Diagramm RLC-System mit $22 \Omega$ Widerstand}
  \label{aufgabe3_bode1}
\end{figure}
\begin{figure}[H]
  \centering
  \includegraphics[width=150mm]{bode_filter03_180o.PNG}
  \caption{Bode-Diagramm RLC-System mit $180 \Omega$ Widerstand}
\end{figure}
\begin{figure}[H]
  \centering
  \includegraphics[width=150mm]{bode_filter03_1ko.PNG}
  \caption{Bode-Diagramm RLC-System mit $1 k\Omega$ Widerstand}
\end{figure}

\noindent In der Schaltung aus Abbildung ~\ref{aufgabe3_schaltplan} werden RL und RC Tiefpassfilter kombiniert und man erh\"ahlt einen Tiefpass 2. Ordnung. Dadurch f\"allt die Spannung oberhalb der Grenzfrequenz schneller ab, im Vergleich zu einem Tiefpass 1. Ordnung. Bei der Schaltung mit einem Widerstand von $22 \Omega$ ist eine Spannung\"uberh\"ohung zu erkennen. Dieser Effekt entsteht im Resonanzfall von Eingangsspannung und Eigenfrequenz des RLC-Systems. Die Bodediagramme unterscheiden sich in der Flankensteilheit der Amplituden- und Phaseng\"ange. Die Grenzfrequenz des Kondensators wird bei gro\ss en Widerstand fr\"uher erreicht ($f_c = \frac{1}{2\pi*R*C}$). Die Grenzfrequenz der Spule hingegen wird bei gro\ss en Widerstand erst sp\"ater erreicht ($f_l = \frac{R}{2\pi*L}$). Das bewirkt, dass bei gro\ss em Widerstand die D\"ampfung schon bei geringerer Frequenz einsetzt, jedoch ist die Flankensteilheit bis zur Grenzfrequenz der Spule geringer.


\end{document}
